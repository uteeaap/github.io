\documentclass[12pt]{article}
\usepackage[utf8]{inputenc}
\usepackage{fullpage}
\usepackage{amsmath}
\usepackage{chemfig}
\usepackage{graphicx}
\usepackage{fancyhdr}
 
 
\title{CHE 153 General Chemistry I Laboratory Manual}
\author{University of Tampa Department of Chemistry, Biochemistry, \& Physics}
%\date{today}
\graphicspath{{images/}}
 
\begin{document}
\maketitle
\newpage
\begin{figure}[htp]
    \centering
    \includegraphics[width=15cm]{nist_periodictable_july2019.pdf}
    \caption{NIST Periodic Table}
    \label{fig:NIST periodic table}
\end{figure}
\newpage
\section{\large Table of Contents}
\begin{center}
\begin{tabular}{ l }
Chemistry Department Rules\\
Math Review\\
\end{tabular}
\end{center}
%\tableofcontents
\begin{center}
\begin{tabular}{ l l } 
 Lab 1 & Density and Uncertainty in Measurements \\ 
 Lab 2 & Separation of the Components of a Mixture \\ 
 Lab 3 & Formula of a Hydrate \\
 Lab 4 & Chemistry of Copper\\ 
 Lab 5 & Analysis of Vinegar\\ 
 Lab 6 & Calorimetry\\
 Lab 7 & Qualitative Analysis of Group I Cations\\ 
 Lab 8 & Qualitative Analysis of Group III Cations\\ 
 Lab 9 & Qualitative Analysis of a General Unknown\\
 Lab 10 & VSEPR and Molecular Modeling\\
\end{tabular}
\end{center}
\begin{center}
\begin{tabular}{ l l } 
 Appendix I & Water Solubility of Inorganic Salts\\ 
 Appendix II 2 & Summary of Solubility Properties of Ions and Solids\\ 
 Appendix III & Common Reagents and Complex Ions in Qualitative Analysis\\
 Appendix IV & Properties of Group I, II, and III Cations\\ 
 Appendix V & Significant Figures and Rounding\\
\end{tabular}
\end{center}
\footnote{Rev 2018.07}{r}
\newpage
This page intentionally left blank
\newpage
\section{Chemistry Department Laboratory Rules}
\textbf{ EYE PROTECTION:} ANSI or OSHA approved protective eyewear must be worn over the eyes at all times in the laboratory, whether you wear prescription glasses or not. Contact lenses should not be worn in the lab because chemical exposure to an eye may cause a contact lens to become permanently fused to the eye. As it is difficult to determine whether someone is wearing contact lenses or not, it is the responsibility of each student to make sure this rule is obeyed.\\
\newline %line break
\textbf{HAND PROTECTION:} No glove protects you from all potential hazards. Latex-free gloves are available in the lab and you are encouraged to use them. When conducting certain experiments, your instructor may require that you wear gloves. No glove should be worn for more than 2 hours without being replaced. Never re-use gloves; always properly dispose of gloves after use. Always remove your gloves when you leave the lab or before you a touch a door knob, computer, telephone, or personal item.\\
\newline %line break
\textbf{ FOOTWEAR:} Shoes must be worn at all times during the laboratory. Your feet must be adequately covered; therefore sandals and open-toed shoes are not acceptable.\\
\newline %line break
\textbf{ CLOTHING:} Your clothing must cover your body from ankle to shoulder, with no exposed skin, and you must wear a short or long sleeved shirt. Shorts, skirts, halters, tank tops, leggings, and exposed midriffs are prohibited. Wearing a lab coat is required and wearing socks is recommended.\\
\newline %line break
\textbf{HAIR:}If your hair is long enough to interfere in the experiment, it must be tied back.\\
\newline %line break
\textbf{ SMOKING:} Smoking in or near the laboratories is prohibited at all times.\\
\newline %line break
\textbf{ FOOD:} Eating, drinking, and chewing gum are prohibited in the laboratories at all times. Food and drink should not be brought into the laboratories.\\
\newline %line break
\textbf{OPEN FLAMES:} Open flames of any type (burners, lighters, matches, etc.) are prohibited in the laboratory, unless specific permission is granted by the instructor.\\
\newline %line break
\textbf{ SCHEDULE:} Students will work only during their scheduled laboratory periods and never alone or unsupervised.\\
\newline %line break
\textbf{ EMERGENCY EQUIPMENT:} Know the location and use of all safety equipment (eyewash fountain, safety shower, fire extinguishers, first aid kit, and safety solutions).\\
\newline %line break
\textbf{ EVACUATION ROUTES:} Know all evacuation routes from your laboratory, including the emergency exits.\\
\newline %line break
\textbf{ACCIDENTS:} All accidents, even minor ones, must be reported to the instructor immediately. \\
\newline %line break
\textbf{CHEMICALS:}\\
\newline %line break
• Never taste or smell any laboratory chemicals.\\
• Never pipet with your mouth.\\
• Never handle or pour laboratory chemicals directly on your hands or body; however, if you do, immediately flush liberally with water and ask for further instructions from your instructor.\\
• Use laboratory chemicals or conduct experiments that generate harmful vapors only in the fume hood\\
\newline %line break
\textbf{REAGENTS:} Return reagents and chemical bottles to their proper places after using. Never place anything into a reagent bottle; pour the reagent out of the bottle or use the dispensing device provided. Never pour anything back into a reagent bottle. Label all secondary containers with the contents and the date filled. \\
\newline
\newline %line break
Student initials: \noindent\rule{4cm}{0.4pt}
\newpage
\noindent\textbf{ WASTE DISPOSAL: }Excess and used materials should be discarded as noted by your instructor, often in specific waste containers. When in doubt, check with your instructor.\\
\newline %line break
\textbf{ WORK SPACE:} Keep your working space neat at all times and clean up this area when you leave for the day.\\
\newline %line break
\textbf{ BEHAVIOR:} Unsafe behavior is not permitted in the lab at any time. Students who present a safety risk to themselves or to others will be asked to leave the laboratory.\\
\newline %line break
\textbf{PERSONAL ELECTRONIC DEVICES:} Use of these is prohibited during laboratory experiments, except as allowed by your instructor.\\
\newline %line break
\textbf{SAFETY INSTRUCTIONS AND INFORMATION:} Consult laboratory manuals, Material Safety Data Sheets\\
\newline %line break
\textbf{(MSDS)}and safety posters in the labs for explanation of the safety rules. One resource for information is the UT Chemical Safety web page, http://utweb.ut.edu/chemicalsafety/. It includes links to MSDS and UT’s Chemical
Hygiene Plan.\\
\newline %line break
\textbf{PREGNANCY:} Because of the inherent chemical hazards present in a lab, if you are or may be pregnant, or are planning to become pregnant during this course, it is your responsibility to advise your instructor and the Chemical Environmental Health and Safety Coordinator (CEHSC), Dr. Stephen Kucera. The instructor and CEHSC will determine whether reasonable accommodations are possible in the lab which can be made to protect the fetus. Every effort will be made to keep this information confidential.\\
\noindent\makebox[\linewidth]{\rule{\paperwidth}{0.4pt}}

By signing below, I acknowledge that:
\begin{itemize}
  \item my instructor has discussed these rules and that I have read and understand these rules, as well as the safety instructions in the laboratory manual.
  \item I understand that if I violate any of these rules, I do so at my own risk. 
  \item I understand that I must conduct myself in a manner that is appropriate for a lab in order to maintain my personal safety, as well as the safety of all other persons in the lab.
  \item I also understand that if I fail to comply with these rules, at the sole discretion of the instructor, I may be expelled from the laboratory for the laboratory period and receive a grade of zero on the experiment that was being conducted. A lab from which you are expelled can not be made up. 
\end{itemize}
Student Name: \noindent\rule{4cm}{0.4pt} Student ID No.: \noindent\rule{3cm}{0.4pt}\\
Signature: \noindent\rule{5cm}{0.4pt} Date: \noindent\rule{4cm}{0.4pt}\\
%\vspace{50mm} %50mm vertical space
%\vfill
\cfoot[CE,CO]{CHEM LAB RULES v1.5.6.doc Rev 08/06;01/07;01/08(Format);0807 (Image);0808(typo);0312;0712}
%need spacing between initials and new page
\newpage
\noindent\textbf{MATH REVIEW}\space\space\space\space\space Name: \rule{5cm}{0.4pt}\\
\noindent\makebox[\linewidth]{\rule{\paperwidth}{0.4pt}}
\emph{Use Chapter 1 in your lecture textbook to answer the following questions.}\\
%\begin{center}
\textbf{Units of Measurement}\\
\label{table:1}
\setlength{\linewidth}{3cm} 1. Give the abbreviation and name of the SI unit for:\\
\begin{tabular}{ l l l l } 
a & time & \noindent\rule{1cm}{0.4pt} & \noindent\rule{2cm}{0.4pt}\\ 
b & temperature & \noindent\rule{1cm}{0.4pt} & \noindent\rule{2cm}{0.4pt}\\ 
c & length & \noindent\rule{1cm}{0.4pt} & \noindent\rule{2cm}{0.4pt}\\
d & electric current & \noindent\rule{1cm}{0.4pt} & \noindent\rule{2cm}{0.4pt}\\ 
e & mass & \noindent\rule{1cm}{0.4pt} & \noindent\rule{2cm}{0.4pt}\\
\end{tabular}
%\end{center}
%\textbf{Units of Measurement}\\
\label{table:2}
%2. Give the abbreviation and name of the SI unit for:\\
\begin{tabular}{ l l l  } 
a & 3780000000 & \noindent\rule{2cm}{0.4pt}\\ 
b & 0.088625 & \noindent\rule{2cm}{0.4pt}\\ 
c & 631900  & \noindent\rule{2cm}{0.4pt}\\
d & $0.00425$ x $10^1$ & \noindent\rule{2cm}{0.4pt}\\ 
e & $514.0$ x $10^-8$ & \noindent\rule{2cm}{0.4pt}\\
\end{tabular}
%Need superscript for items d and e
%need to represent multiply with x for item d and e
\newline %line break
Give the \textbf{abbreviation} and \textbf{name} of the SI prefix \\
\label{table:1}
\setlength{\linewidth}{3cm} 1. Give the abbreviation and name of the SI unit for:\\
\begin{tabular}{ l l l l } 
a & $10^-2$ & \noindent\rule{1cm}{0.4pt}  & \noindent\rule{2cm}{0.4pt}\\ 
b & $10^-6$ & \noindent\rule{1cm}{0.4pt} & \noindent\rule{2cm}{0.4pt}\\ 
c & $10^-9$  & \noindent\rule{1cm}{0.4pt} & \noindent\rule{2cm}{0.4pt}\\
d & $10^-12$ & \noindent\rule{1cm}{0.4pt} & \noindent\rule{2cm}{0.4pt}\\ 
e & $10^3$ & \noindent\rule{1cm}{0.4pt}  & \noindent\rule{2cm}{0.4pt}\\ 
f & $10^-1$ & \noindent\rule{1cm}{0.4pt} & \noindent\rule{2cm}{0.4pt}\\ 
g & $10^-3$  & \noindent\rule{1cm}{0.4pt} & \noindent\rule{2cm}{0.4pt}\\
h & $10^-15$ & \noindent\rule{1cm}{0.4pt} & \noindent\rule{2cm}{0.4pt}\\ 
\end{tabular}
\end{document}
