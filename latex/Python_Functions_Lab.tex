\documentclass[]{article}
\usepackage{listings}
\usepackage{testphys}
\usepackage{amsmath}
\usepackage{parskip}
\usepackage{graphicx}

\font\btt=rm-lmtk10
\usepackage{color}
 
\definecolor{codegreen}{rgb}{0,0.6,0}
\definecolor{codegray}{rgb}{0.5,0.5,0.5}
\definecolor{codepurple}{rgb}{0.58,0,0.82}
\definecolor{backcolour}{rgb}{0.95,0.95,0.92}
 
\lstdefinestyle{texstyle}{
    backgroundcolor=\color{backcolour},   
    commentstyle=\color{codegreen}\itshape\small,
    keywordstyle=\color{black}\btt,
    numberstyle=\tiny\color{codegray},
    stringstyle=\color{codepurple},
    basicstyle=\ttfamily\small,
    breakatwhitespace=false,         
    breaklines=true,                 
    captionpos=b,                    
    keepspaces=true,                 
    numbers=left,                    
    numbersep=5pt,                  
    showspaces=false,                
    showstringspaces=false,
    showtabs=false,                  
    tabsize=2,
    language=Python
}
\lstset{style=texstyle}

\renewcommand{\_}{\char`_}
\newcommand{\bs}{\symbol{92}}

\begin{document}

\pagestyle{empty}

\noindent
{\Large\bf PHY 365}
\vspace{4pt}

\noindent
{\Large\bf Creating Functions} \hfill {\large Name:} \hrulefill
\vspace{6pt}

%\noindent
%{\Large\bf Do your best!}
\vspace{11pt}

\noindent This is a quick Python refresher for developing functions. Thanks to SoftwareCarpentry for the tutorial structure. 

\medskip\hrule

\medskip
Before we get started with today's project: 
\begin{enumerate}
\item Make sure that SolarSystem is mounted on the Mac. Remember: \texttt{Finder->Go->Connect to Server. }
\item Load \textit{Spyder}. In the upper right corner, you should see a black ``open folder'' icon. Click that. Navigate using Finder to your home directory on SolarSystem (If this makes no sense as an instruction, ask!) and select ``Open''. This will set your working directory to your home directory. If you did this right, you should see the directory that you are in as \texttt{\bs Volumes\bs home}.   
\end{enumerate}

\section{What is a function?}

A basic Python function is functionally (pun intended) the same as a mathematical function; that is, it takes some kind of input and gives some kind of output. Here is an example of the most basic function, the test program ``Hello World''. 
\begin{lstlisting}
def hello():
    print('Hello World!')
\end{lstlisting}
Before we use this function, let's break it down. 
\begin{itemize}
\item \texttt{def} is the command to begin the \textbf{def}inition a function. 
\item \texttt{hello()} is the name of the function and the list of inputs into the function. Since this function takes no inputs, the parentheses are empty. 
\item \texttt{:} A colon is necessary to start the function. Its purpose is almost grammatically literal! You can read the function in English like this:\\[1ex] 
This is the definition of a function named hello that requires no inputs: \\
print the string 'hello world'

\item \texttt{spaces} all functions are indented by 4 spaces. This is how Python knows what is inside the function and what is not inside the function. The console will interpret the colon immediately, and continue automatically with the symbol `\texttt{...:}', but you still have to indent. If you are writing a script, you need to provide indents yourself too. 
\end{itemize}

How do you run this function? In the console, type: 
\begin{lstlisting}
def hello():
    print('Hello World!')

hello()
\end{lstlisting}
Always leave a blank line after the definition. The interpreter knows that this indicates the end of a function. When you call the function, don't forget the parentheses, otherwise python will think you're talking about a variable named \texttt{hello}.

\newpage

\subsection{Parameters}

Consider the mathematical function $$f(x,y) = 2y-x$$ This function has two parameters, $x$ and $y$. If I were to say that $x=3$ and $y=4$, then we could write: 
\begin{align*}
f(x,y) &= 2y-x\\
f(3,4) &= 2(4)-3\\
&=5
\end{align*}
In Python, the \textit{parameters} 3 and 4 are said to be \textit{passed} into the function. Parameters don't have to be numbers. They can also be strings, lists, NumPy arrays, etc. Look at this function: 
\begin{lstlisting}
def print_date(year, month, day):
    joined = str(year) + '/' + str(month) + '/' + str(day)
    print(joined)

print_date(1871, 3, 19)
\end{lstlisting}
What is the output of this function? What if you change the order of the numbers in the function call? Is order of inputs important? Try \texttt{print\_ date(month=3, day=19, year=1871)} for fun! 

\medskip\makebox[0.96\textwidth]{\hrulefill}

\medskip\makebox[0.96\textwidth]{\hrulefill}

\medskip\makebox[0.96\textwidth]{\hrulefill}

\subsection{Function Returns}

All functions \textit{return} a value. The two previous examples simply print a statement, and are done; but they still return something. In these cases, they return the value \texttt{None}, which is exactly what it sounds like. This can create super subtle bugs in your code:
\begin{lstlisting}
result = print_date(1871, 3, 19)
print('result of call is:', result)
\end{lstlisting}
What is going on here? Hint: What is \texttt{result}? 

\medskip\makebox[0.96\textwidth]{\hrulefill}

\medskip\makebox[0.96\textwidth]{\hrulefill}

\medskip\makebox[0.96\textwidth]{\hrulefill}

For a function to return a value, we need to explicitly tell it to, using the \texttt{return} statement. The following program takes in a \textit{list} as a parameter and returns the average value of the list's elements. 
\begin{lstlisting}
def average(values):
    if len(values) == 0:
        return None
    return sum(values) / len(values)

average([1,2,3])
average([])
average(1,2,3)
\end{lstlisting}
Explain what happens when you run the function with those inputs. 

\medskip\makebox[0.96\textwidth]{\hrulefill}

\medskip\makebox[0.96\textwidth]{\hrulefill}

\medskip\makebox[0.96\textwidth]{\hrulefill}


\section{Documentation}
When you create a function, you should always ad in a description of what the function does. This is not just good programming practice, it's incredibly helpful when you put a block of code on the side for a few months and then come back to it\dots and forget what it was supposed to do. 

There are two kinds of documentations. A comment is a short snippet of text to clarify a bit of code. It is always separated from the code by a hashtag: \texttt{\#}. Comments can appear anywhere in the code, and are ignored by the interpreter. 
\begin{lstlisting}
def average(values):
    if len(values) == 0: # This captures empty strings to prevent errors
        return None # Probably needs a better error message?
    return sum(values) / len(values)

average([1,2,3])
average([])
average(1,2,3)
\end{lstlisting}

A \textit{docstring} is separated from the code by three quotes (single or double): \texttt{"""}. In a function, a docstring can be used as the "help menu" for the function. Try it! 
\begin{lstlisting}
def std_dev(sample):
    """
    This function takes in a list or array of numbers and finds the standard 
    deviation of the list.
    """ 
    sample_sum = 0
    for value in sample:
        sample_sum += value

    sample_mean = sample_sum / len(sample)

    sum_squared_devs = 0
    for value in sample:
        sum_squared_devs += (value - sample_mean) * (value - sample_mean)

    return numpy.sqrt(sum_squared_devs / (len(sample) - 1))

data = [2,4,7,6,9,3,4,2,5,3,6]

help(std_dev)

print(std_dev(data))
\end{lstlisting}

\newpage

\section{Questions}

\bbq
\bq
Using what you know, fill in the blanks to create a function that takes a list of numbers as an argument and returns the first negative value in the list. What does your function do if the list is empty? What should it do?\eq
\begin{lstlisting}
def first_negative(values):
    for v in ____:
        if ____:
            return ____
\end{lstlisting}

\bq 
The code below will run on a label-printer for chicken eggs. A digital scale will report a chicken egg mass (in grams) to the computer and then the computer will print a label.
\begin{lstlisting}
 import random
 for i in range(10):

    # simulating the mass of a chicken egg
    # the (random) mass will be 70 +/- 20 grams
    mass=70+20.0*(2.0*random.random()-1.0)

    print(mass)
   
    #egg sizing machinery prints a label
    if(mass>=85):
       label='jumbo'
    elif(mass>=70):
       label='large'
    elif(mass<70 and mass>=55):
       label='medium'
    else:
       label='small'
    
    print(label)

\end{lstlisting}
Create a function definition for \texttt{print\_egg\_label()} that will simplify the program above so that it looks like this: 
\begin{lstlisting}
 # revised version
 import random
 for i in range(10):

    # simulating the mass of a chicken egg
    # the (random) mass will be 70 +/- 20 grams
    mass=70+20.0*(2.0*random.random()-1.0)

    print('Mass={}g, this egg is {}'.format(mass,print_egg_label(mass))   
\end{lstlisting}
An oversize egg will have a mass of more than 91 grams, and a spoiled or broken egg will probably have a mass that’s less than 50 grams. Modify your \texttt{print\_egg\_label()} function to account for these error conditions.\eq

\newpage

\bq The function known as the \textit{logistic map} has many interesting solutions. We will investigate this function later in class, but it's fun to play with. The logistic map is given by: 
\begin{equation*}
x_{n+1} = \mu x_n(1-x_n)
\end{equation*}
For a given value of $\mu\in (0,4)$, you start with any value of $x_n\in (0,1)$, and solve for $x_{n+1}$. This value of $x_{n+1}$ becomes the new value of $x_n$ and the process is repeated. After many iterations, there are one of three outcomes: 
\begin{enumerate}
\item The value settles down to a single fixed number and does not change. This is called a \textit{fixed point}. 
\item The value bounces periodically between a finite number of values, say two or four. This is called a \textit{limit cycle}. 
\item The value does not repeat, running through a seemingly random set of numbers. This is \textit{deterministic chaos}---the numbers look ``chaotic'' and random, but they are clearly predictable from the equation. 
\end{enumerate}
Write a function that will iterate the logistic map for a thousand steps given $\mu=2$ and some initial value $x_1$, and plot the results. The plot should be properly labeled as ``Index'' on the horizontal and ``$x$'' on the vertical. Find some values of $x_1$ that lead to fixed points, limit cycles and chaos. 



\eq
\eeq
\end{document}